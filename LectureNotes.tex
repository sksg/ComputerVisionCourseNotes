\documentclass{lecturenotes-handout}

\usepackage{lecturenotes-xtitle}
\title[Computer Vision Lecture Notes; A practical guide to multi-view geometry]%
      {Computer Vision Course Notes\\--- A practical guide to multi-view geometry}
\author[Søren K. S. Gregersen (sorgre@dtu.dk), DTU Compute]%
       {Søren K. S. Gregersen (\href{mailto:sorgre@dtu.dk}{sorgre@dtu.dk}), DTU Compute}

% \usepackage{hyphenat}
\addbibresource{references.bib}

\usepackage{lecturenotes-xmath}
\DeclareMathOperator{\HOM}{HOM}
\DeclareMathOperator{\INHOM}{INHOM}

\usepackage{lecturenotes-boxes}


\begin{document}

\maketitle

\tableofcontents
\hrule

\section{Homogeneous coordinates and projective space}
Computer vision and computational imaging mathematics are often heavy with complicated expressions. Any tricks that can simplify such expressions are appreciated a lot down the line. Homogeneous coordinates are such a trick. Rotations, translations, and perspective transforms of entire point sets\remark{\includestandalone[mode=buildnew,width=\linewidth]{observer-transformations}Imagine observing, for example, the \emph{Stanford Bunny}\footnotemark\ from different points of view. The mathematics may involve world-to-observer transformations and observer-to-observer transformations, as well as perspective transformations into the eyes of the observers. In homogeneous coordinates, these are single matrix-vector operations.}\footcitetext{turk1994} are expressed much easier---especially with all transforms at once. The homogeneous coordinates extend the standard Cartesian coordinates by an extra \emph{scale dimension} \(w\) as a separate coordinate. Two examples are
\begin{align}
  \vect{q} = \begin{bmatrix} w x \\ w y \\ w \end{bmatrix} \AND
  \vect{Q} = \begin{bmatrix} w x \\ w y \\ w z \\ w \end{bmatrix}
\end{align}
describing two-dimensional (2D) and three-dimensional (3D) points, respectively. In other words, any \(N\)-dimensional vector in the Cartesian coordinate system effectively becomes an \((N+1)\)-dimensional vector in homogeneous coordinates.

We define the homogeneous transformation (\( \HOM \)) of a point from inhomogeneous to homogeneous coordinates
\begin{align}
    \HOM(\vect{v}) = \begin{bmatrix} w \vect{v} \\ w \end{bmatrix} = \vect{q} \,, \WHERE w \neq 0. \label{eq:hom.def}
\end{align}
Here \(\vect{v}\) and \(\vect{q}\) are the inhomogeneous and homogeneous coordinates, respectively. Now \(\vect{v}\) and \(\vect{q}\) describe the same point in different coordinates i.e. \emph{they are equivalent}. Notice that the scaling dimension \(w\) could be arbitrarily chosen, say, to \(w = 3.14\) and the points would still be equivalent, as long as we maintain \(w \neq 0\). For convenience, the scale \(w = 1\) is often chosen in the above \( \HOM \) transformation.\remark{\emph{What does it mean to scale homogeneous coordinates by \(w\)?} Well, just think of homogeneous coordinates as a mathematical trick. It makes point set (rigid) transformations easier and it linearizes the perspective projection transformation. In effect, all these operations can be condensed into a single matrix with operates on the homogeneous vector.}

\begin{example}\label{exp:homogeneous-coords}%
  Consider the 2D point
  \begin{align}
      \vect{v} = \begin{bmatrix} 3 \\ 4 \end{bmatrix} \,.
  \end{align}
  Any of the three following homogeneous points are equivalent to \(\vect{v}\)
  \begin{align}
      \vect{q}_1 = \begin{bmatrix} 1 * 3 \\ 1 * 4 \\ 1 \end{bmatrix} \,, \quad
      \vect{q}_2 = \begin{bmatrix} 2 * 3 \\ 2 * 4 \\ 2 \end{bmatrix} \,, \AND
      \vect{q}_3 = \begin{bmatrix} \pi/6 * 3 \\ \pi/6 * 4 \\ \pi/6 \end{bmatrix}
  \end{align}

  The homogeneous coordinates are just scaled with respect to each other. You could also say that equivalent points in homogeneous coordinates are, in fact, parallel vectors.
\end{example}

Getting back to the Cartesian coordinates we have the (reverse) inhomogeneous transformation (\( \INHOM \)). By reversing \cref{eq:hom.def} the \( \INHOM \) transform is defined as
\begin{align}
  \INHOM (\vect{q}) = \vect{v} / w \,, \WHERE w \neq 0. \label{eq:inhom.def}
\end{align}
Here it is obvious to see that we cannot have \(w = 0\) since we divide by \(w\). A point with \(w = 0\) is no longer a point in real space. \remark{This changes, however, when we expand our homogeneous coordinates to also express differences vectors i.e. vectors between points such as \(\vect{\delta} = \vect{v}_1 - \vect{v}_2\). We will dive more into this in \cref{sec:homogeneous-coords-vector-displacements}.}

\begin{exercise}\label{exc:homogeneous-coords}%
  Which of the following homogeneous coordinates are equivalent:
  \begin{align}
      \vect{q}_1 = \begin{bmatrix} 1    \\  2   \\  2.5  \\  1 \end{bmatrix} \,, \;
      \vect{q}_2 = \begin{bmatrix} 3.4  \\  6.8 \\  8.4  \\  3.4 \end{bmatrix} \,, \;
      \vect{q}_3 = \begin{bmatrix} 5.1  \\ 10.2 \\ 12.75 \\  5.1 \end{bmatrix} \,, \AND
      \vect{q}_4 = \begin{bmatrix} 7    \\ 14.4 \\ 18.   \\  7.2 \end{bmatrix} \,?
  \end{align}
  What are the inhomogeneous coordinates?

  \begin{scorecard}[Scores:][How did you solve this exercise?]
    \item Using pen and paper \( \rightarrow \) old school student,
    \item mathematical software \( \rightarrow \) modern millennial, or
    \item thinking really hard \( \rightarrow \) natural born mathematician.
  \end{scorecard}

\solution
  The inhomogeneous coordinates \(\vect{v}_i = \INHOM(\vect{q}_i)\) are:
  \begin{align}
      \vect{v}_1 = \begin{bmatrix} 1    \\  2   \\  2.5  \end{bmatrix} \,, \;
      \vect{v}_2 = \begin{bmatrix} 1    \\  2   \\  2.47 \end{bmatrix} \,, \;
      \vect{v}_3 = \begin{bmatrix} 1    \\  2   \\  2.5  \end{bmatrix} \,, \AND
      \vect{v}_4 = \begin{bmatrix} 0.97 \\  2   \\  2.5  \end{bmatrix} \,.
  \end{align}
  Evidently only \(\vect{v}_1 = \vect{v}_3\) such that \(\vect{q}_1\) and \(\vect{q}_3\) are equivalent.
\end{exercise}

\subsection{Vector displacements versus points}\label{sec:homogeneous-coords-vector-displacements}

While the homogeneous coordinates are smart and easy to use, they do require a change of mindset. Adding and subtracting coordinates does not work like we necessarily expect. Compare these two cases of adding two points using, respectively, Cartesian and homogeneous coordinates:
\begin{align}
\vect{v}_3 = \vect{v}_1 + \vect{v}_2 \AND \vect{q}_3 = \vect{q}_1 + \vect{q}_2 \,.
\end{align}
If we insert the \(\HOM\) transformations
\begin{align}
\vect{q}_1 = \begin{bmatrix}\vect{v}_1 \\ w_1 \end{bmatrix} \AND \vect{q}_2 = \begin{bmatrix}\vect{v}_2 \\ w_2 \end{bmatrix} \,,
\end{align}
then adding them yields
\begin{align}
\vect{q}_3 = \begin{bmatrix}\vect{v}_1 \\ w_1 \end{bmatrix} + \begin{bmatrix}\vect{v}_2 \\ w_2 \end{bmatrix} = \begin{bmatrix}\vect{v}_3 \\ w_1 + w_2 \end{bmatrix} \,.
\end{align}
Notice that now \(\INHOM(\vect{q}_3) = \vect{v}_3 / (w_1 + w_2) \neq \vect{v}_3 \forall w_1, w_2 \in \mathbb{R} \).\remark{If you are unfamiliar with the \emph{forall} symbol \(\forall\) then, for example, \(A \forall B\) means that \(A\) is true for all elements of \(B\). In this case \(\vect{v}_3 / (w_1 + w_2) \neq \vect{v}_3 \forall w_1, w_2 \in \mathbb{R}\) means that \(\vect{v}_3 / (w_1 + w_2)\) is not equal to \(\vect{v}_3\) for all cases of \(w_1 \in \mathbb{R}\) and \(w_2 \in \mathbb{R}\). In fact, they are only equal in the case that \(w_1 + w_2 = 1\)} In other words, adding two homogeneous points \emph{is not equivalent to} adding the same inhomogeneous points. This seemingly breaks consistency between homogeneous and inhomogeneous coordinates. To fix this we introduce displacements (\(\vect{\delta}\)), which are different from points (\(\vect{v}\)).

A point \(\vect{v}\) and can be displaced by adding or subtracting any number of displacements, like so:
\begin{align}
\vect{v}_2 = \vect{v}_1 + \vect{\delta}_1 + \vect{\delta}_2 + \dots \,.
\end{align}
A displacement can also be a superposition of other displacements, like so
\begin{align}
\vect{\delta}_n = \vect{\delta}_1 + \vect{\delta}_2 + \vect{\delta}_3 + \dots \,.
\end{align}
This makes sense in the real world too. Adding two points together has no physical meaning; but displacing points does have meaning. Moreover, points cannot suddenly become displacements and vice versa.

The (in)homogeneous transformations of displacements is different from points [compare to \cref{eq:hom.def,eq:inhom.def}]. The homogeneous transform of displacements is defined
\begin{align}
    \HOM(\vect{\delta}) = \begin{bmatrix} \vect{\delta} \\ 0 \end{bmatrix} = \vect{\xi} \,,
\end{align}
where the homogeneous scale is zero. Likewise the inhomogeneous transform of displacements is defined
\begin{align}
    \INHOM (\vect{\xi}) = \INHOM \begin{bmatrix} \vect{\delta} \\ 0 \end{bmatrix} = \vect{\delta}
\end{align}

This is, in fact, a powerful tool we suddenly have at our disposal. In traditional Cartesian representation, there is no such distinction between points and displacements---they are all just vectors of the same length. In homogeneous coordinates, however, they are easy to distinguish; points have \(w \neq 0\) and displacements have \(w = 0\). You just have to look at the last dimension and you know exactly whether it is a point or a displacement.\remark{You can also think of displacements as being points infinitely far away---because the Cartesian point vector gets divided by zero. In effect, displament vectors are too far from the origin \(\vect{0}\) to be represented as real points. These vectors are very useful in computer science. For example, in graphics for representing unidirectional lights or surface normals because they are not relative to an origin.}

\begin{example}
    Consider the displacement \(\vect{\Delta}\) of a 3D point \(\vect{V}\), where
    \begin{align}
      \vect{V} = \begin{bmatrix} 4 \\ 2 \\ 9\end{bmatrix} \AND
      \vect{\Delta} = \begin{bmatrix} 1 \\ 0 \\ -7\end{bmatrix} \,.
    \end{align}
    In inhomogeneous coordinates the final point is \(\vect{U} = \vect{V} + \vect{\Delta}\). The homogeneous coordinates are defined
    \begin{align}
        \vect{Q}_V = \HOM (\vect{V}) = \begin{bmatrix} 4 \\ 2 \\ 9 \\ 1\end{bmatrix} \AND
        \vect{\xi}_\Delta = \HOM (\vect{\Delta}) = \begin{bmatrix} 1 \\ 0 \\ -7 \\ 0\end{bmatrix} \,.
    \end{align}
    Using homogeneous coordinates the final point is
    \begin{align}
    \vect{Q}_U = \vect{Q}_V + \vect{\xi}_\Delta
               = \begin{bmatrix} 4 \\ 2 \\ 9 \\ 1\end{bmatrix} + \begin{bmatrix} 1 \\ 0 \\ -7 \\ 0\end{bmatrix}
               = \begin{bmatrix} 5 \\ 2 \\ 2 \\ 1\end{bmatrix} = \begin{bmatrix} \vect{U} \\ 1\end{bmatrix}\,.
    \end{align}

    With the distinction between points and displacements it does not matter whether we work in homogeneous or inhomogeneous representation. The resulting point is always a simple addition.
\end{example}

\subsection{Representing 2D lines and 3D planes}

Homogeneous coordinates can also be used to represent lines and planes. Consider the 2D line given by the line equation
\begin{align}
0 = a x + b y + c \,,
\end{align}
where \(a\), \(b\), and \(c\) represent the line, and \(x\) and \(y\) are the coordinates of points on the line. By rearrangement, the line equation can be represented in vector form
\begin{align}
0 = \begin{bmatrix}a & b & c\end{bmatrix} \begin{bmatrix}x \\ y \\ 1\end{bmatrix} \label{eq:line.eq}
  = \vect{l}^{\TR} \vect{q}
\end{align}
where \(\vect{l}^{\TR} = [\begin{matrix}a & b & c\end{matrix}]\) represents the line and \(\vect{q}\) represents coordinates of the points on the line. Along the same lines---no pun intended---we can represent 3D planes from the plane equation
\begin{align}
0 = a x + b y + c z + d
  = \begin{bmatrix}a & b & c & d\end{bmatrix} \begin{bmatrix}x \\ y \\ z \\ 1\end{bmatrix} \label{eq:plane.eq}
  = \vect{P}^{\TR} \vect{Q} \,,
\end{align}
where \(\vect{P}^{\TR} = [\begin{matrix}a & b & c & d\end{matrix}]\) represents the plane and \(\vect{Q}\) is the homogeneous coordinates of the points in the plane. Notice in particular that the above equations do not rely on \(w = 1\). In fact, you could scale both sides of the equations with any \(w\) and the above equations still hold. Of course setting \(w = 0\) would only return the trivial and uninteresting case of \(0 = 0\).

It is worth pointing out that the homogeneous 2D line representation makes line intersections easy to derive. Given two 2D lines \(\vect{l}_1\) and \(\vect{l}_2\) we have the relations
\begin{align}
0 = \vect{l}_1^{\TR} \vect{q} \AND
0 = \vect{l}_2^{\TR} \vect{q} \label{eq:hom.lineintersect} \,,
\end{align}
where \(\vect{q}\) is the intersection point. A solution forms as the cross-product
\begin{align}
\vect{q} = \vect{l}_1 \times \vect{l}_2 \,,
\end{align}
by assuming \(\vect{l}_1\), \(\vect{l}_2\), and \(\vect{q}\) are just ordinary vectors of three dimensions (3-vectors). This solution is easily validated by inserting into \cref{eq:hom.lineintersect}.

As it also turns out, using the homogeneous representation makes point-to-line or point-to-plane distance calculations super easy. We define these distances as the shortest distances between the points and the lines or plane, respectively. Then given the line \(\vect{l}\) and the point \(\vect{q}_i\) you find the distance \(d\) between them
\begin{align}
d = \frac{\vect{l}^{\TR} \vect{q}_i}{||\vect{n}||w} \,. \label{eq:hom.dist2line}
\end{align}
The line \(\vect{l}^{\TR} = [\vect{n}^{\TR} \; \alpha]\), the 2D point \(\vect{q}^{\TR} = [\vect{v}^{\TR} \; w]\), and the 2D vector \(\vect{n}\) is perpendicular to the line \(\vect{l}\). The point-to-plane distance \(D\) is given similarly by
\begin{align}
D = \frac{\vect{P}^{\TR} \vect{Q}_i}{||\vect{N}||w} \,, \label{eq:hom.dist2plane}
\end{align}
where the plane \(\vect{P}^{\TR} = [\vect{N}^{\TR} \, \beta]\), the 3D point \(\vect{Q}^{\TR} = [\vect{V}^{\TR} \; w]\), and the 3D vector \(\vect{N}\) is perpendicular to the plane \(\vect{P}\). The proofs of these distance formulas are left as exercise~\ref{exc:homog.dist.formulas} for the reader. In the special cases of \(w = ||\vect{n}|| = 1\) or \(w = ||\vect{N}|| = 1\) in case of, respectively, line or plane distances, these relations simplify to
\begin{align}
d = \vect{l}^{\TR} \vect{q}_i \OR D = \vect{P}^{\TR} \vect{Q}_i \,.
\end{align}

\begin{exercise}\label{exc:homog.dist.formulas}
  Prove the distance formulas. Given a 2D line \(\vect{l} = [\vect{n}^{\TR} \; \alpha]\) and a 3D plane \(\vect{P} = [\vect{N}^{\TR} \, \beta]\), prove \cref{eq:hom.dist2line,eq:hom.dist2plane} using as few hints from below as possible. The average student may need all hints.

  \begin{scorecard}[Exercise hints:]
    \item \textbf{Hint 1}: The attentive reader will have noticed from the main text that the vectors \(\vect{n}\) and \(\vect{N}\) are perpendicular to the line and plane, respectively.
    \item \textbf{Hint 2}: The shortest paths are, in fact, perpendicular to line or plane. In other words, the distance is a projection onto the vectors \(\vect{n}\) and \(\vect{N}\).
    \item \textbf{Hint 3}: The \cref{eq:line.eq,eq:plane.eq} can be used to show that \(\alpha\) and \(\beta\) are the \emph{negative} distances between origin \(\vect{0}\) and the line or plane, respectively.
    \item \textbf{Hint 4}: The shortest distance can be shown to be the projections of the point vectors onto the normal vectors \emph{minus} the distances from origin \(\vect{0}\) to the line or plane.
  \end{scorecard}
\solution
  From the line and plane \cref{eq:line.eq,eq:plane.eq} we know that \(\vect{n}\) and \(\vect{N}\) are perpendicular to the line and plane, respectively. Given the 2D point \(\vect{v}\) and the 3D point \(\vect{V}\) in Cartesian coordinates, the shortest distances are projections onto \(\vect{n}\) and \(\vect{N}\) i.e.
  \begin{align}
    d = \frac{\vect{n} * (\vect{v} - \vect{v}_l)}{||\vect{n}||} \AND
    D = \frac{\vect{N} * (\vect{V} - \vect{V}_P)}{||\vect{n}||} \,,
  \end{align}
  where \(\vect{v}_l \in \{\vect{l}\}\) and \(\vect{V}_P \in \{\vect{P}\}\) are points on the line or plane, respectively. Inserting the points \(\vect{v}_l\) and \(\vect{V}_P\) into \cref{eq:line.eq,eq:plane.eq}, respectively, we find that
  \begin{align}
    \vect{n} * \vect{v}_l = - \alpha \AND
    \vect{N} * \vect{V}_P = - \beta \,.
  \end{align}
  The distances are then simplified into
  \begin{align}
    d = \frac{\vect{n} * \vect{v} + \alpha}{||\vect{n}||} \AND
    D = \frac{\vect{N} * \vect{V} + \beta}{||\vect{n}||} \,,
  \end{align}
  In the usual fashion of homogeneous representations, we rearrange these formulas into
  \begin{align}
    d = \frac{\begin{bmatrix}\vect{n}^{\TR} & \alpha\end{bmatrix} * \begin{bmatrix}\vect{v} \\ 1\end{bmatrix}}{||\vect{n}||} = \frac{\vect{l}^{\TR} * \vect{q}}{||\vect{n}||w} \AND
    D = \frac{\begin{bmatrix}\vect{N}^{\TR} & \beta\end{bmatrix} * \begin{bmatrix}\vect{V} \\ 1\end{bmatrix}}{||\vect{n}||} = \frac{\vect{P}^{\TR} * \vect{Q}}{||\vect{n}||w} \,,
  \end{align}
  where \(\vect{q}^{\TR} = w [\vect{v}^{\TR} \; 1]\) and \(\vect{Q}^{\TR} = w [\vect{v}^{\TR} \; 1]\).
\end{exercise}

\subsection{Homogeneous rigid transformations}

\begin{figure}
  \centering
  \hfill
  \includestandalone[mode=buildnew,height=3cm]{homocoords-rigidtransform-tripod}
  \hfill
  \includestandalone[mode=buildnew,height=3cm]{homocoords-rigidtransform-robot}
  \hfill
  \caption{Schematic illustrations of two example transformations from world to camera (\(\mathcal{W}\rightarrow\mathcal{C}\)); the cases are a simple tripod (left) and a more complicated robotic arm (right).%
  \label{fig:homocoords-rigidtransform}}%
\end{figure}

One of the main attractions of a homogeneous representation is the expression of rigid world transformations---rotations and translations of points in space. This could be, for example, the transformation from world space to camera space, or a more involved example of transformations between each arm of a robotic set up. Both are illustrated in \cref{fig:homocoords-rigidtransform}. The traditional way to change the basis of coordinate system from, say, a point \(\vect{v}_{\mathcal{W}}\) in world space to an equivalent point \(\vect{v}_{\mathcal{C}}\) is
\begin{align}
    \vect{v}_{\mathcal{C}} = \matr{R} \vect{v}_{\mathcal{W}} + \vect{t} \,, \label{eq:inhomo-rigidT-1step}
\end{align}
where \(\matr{R}\) is the rotation matrix and \(\vect{t}\) is the translation vector. This corresponds to the simple single rotation and translation of, for example, a tripod such as illustrated in \cref{fig:homocoords-rigidtransform} (left). If, on the other hand, we consider the a more complicated case such as, for example, the robotic arm illustrated in \cref{fig:homocoords-rigidtransform} (right) the change of basis becomes a little more involved:
\begin{align}
    \vect{v}_{\mathcal{C}} = \matr{R}_4 \left[\matr{R}_3 \left(\matr{R}_2 \left[\matr{R}_1 \vect{v}_{\mathcal{W}} + \vect{t}_1\right] + \vect{t}_2\right) + \vect{t}_3\right] + \vect{t}_4 \,. \label{eq:inhomo-rigidT-mstep-long}
\end{align}
Alternatively, we can formulate a total rotation \(\matr{\Phi}\) and translation \(\vect{\tau}\) where
\begin{align}
    \vect{v}_{\mathcal{C}} &= \matr{\Phi} \vect{v}_{\mathcal{W}} + \vect{\tau} \,, \\
    \matr{\Phi} &= \matr{R}_4 \matr{R}_3 \matr{R}_2 \matr{R}_1 \AND \\
    \vect{\tau} &= \matr{R}_4 \left[\matr{R}_3 \left(\matr{R}_2 \vect{t}_1 + \vect{t}_2\right) + \vect{t}_3\right] + \vect{t}_4\,. \label{eq:inhomo-rigidT-mstep}
\end{align}

However, if we examine \cref{eq:inhomo-rigidT-1step} we may vectorize the expression like
\begin{align}
    \vect{v}_{\mathcal{C}} &= \begin{bmatrix} \matr{R} & \vect{t} \end{bmatrix} * \begin{bmatrix} \vect{v}_{\mathcal{W}} \\ 1 \end{bmatrix} \,, \\
    \vect{q}_{\mathcal{C}} &= \begin{bmatrix} \vect{v}_{\mathcal{C}} \\ 1 \end{bmatrix} = \begin{bmatrix} \matr{R} & \vect{t} \\ \matr{0} & 1 \end{bmatrix} * \begin{bmatrix} \vect{v}_{\mathcal{W}} \\ 1 \end{bmatrix} = \matr{T} \vect{q}_{\mathcal{W}} \,,
\end{align}
where \(\vect{q}_{\mathcal{W}}\) and \(\vect{q}_{\mathcal{C}}\) are the homogeneous representations of \(\vect{v}_{\mathcal{W}}\) and \(\vect{v}_{\mathcal{C}}\). The homogeneous transformation matrix \(\matr{T}\) now does both rotation and translation in one step. Notice that the \(\matr{0} = \begin{bmatrix} 0 & 0 & 0 \end{bmatrix}\) inside \(\matr{T}\) to match the other matrix blocks. This representation makes particularly the book-keeping of multi-step transformations much simpler, such as
\begin{align}
    \vect{q}_{\mathcal{C}} &= \matr{T}_4 \matr{T}_3 \matr{T}_2 \matr{T}_1 \vect{q}_{\mathcal{W}} \,, \label{eq:hom-rigidT-mstep}
\end{align}
in the case of \cref{eq:inhomo-rigidT-mstep}.\remark{Using the homogeneous transformations is very often faster on modern computers. Both \cref{eq:inhomo-rigidT-mstep,eq:hom-rigidT-mstep} allow for precomputation of the total transformations. However, a pure matrix-vector product is an operation that has been heavily optimized for. In other words, the homogeneous equation costs fewer CPU cycles.}


\subsection{Homogeneous perspective transformation}

Perspective transformations are nicely expressed in homogeneous coordinates. The inhomogeneous expression for the 2D projection \(\vect{v}\) of a 3D point \(\vect{V}\) is
\begin{align}
\vect{v} = \begin{bmatrix}x \\ y\end{bmatrix} = \begin{bmatrix}X / Z \\ Y / Z\end{bmatrix} \,, \WHERE
\vect{V} = \begin{bmatrix}X \\ Y \\ Z\end{bmatrix} \,.
\end{align}
However, this appears awfully similar to an \(\INHOM\) transform of the homogeneous point \(\vect{q} = \vect{V}\). In other words, while projecting 3D points into a 2D plane, the depth acts as a homogeneous scaling. In general, the homogeneous projection equation can be stated as
\begin{align}
  \vect{q} = \matr{P} * \vect{Q} \,,
\end{align}
where \(\vect{Q}\) is a 3D point in homogeneous coordinates, \(\matr{P}\) is a \(3\times4\) projection matrix, and \(\vect{q}\) is the 2D point projection of \(\vect{Q}\). Notice that because homogeneous coordinates can be scaled arbitrarily, \(\matr{P}\) only has 11 degrees of freedom.

\subsection{Exercises}

\begin{exercises}

\begin{exercise}
Consider the four two-dimensional (2D) points
\begin{align}
\vect{q}_{0} = \begin{bmatrix} 1 \\ 2 \\ 1 \end{bmatrix} \,, \;
\vect{q}_{1} = \begin{bmatrix} 4 \\ 2 \\ 2 \end{bmatrix} \,, \;
\vect{q}_{2} = \begin{bmatrix} 6 \\ 4 \\ -1 \end{bmatrix} \,, \AND
\vect{q}_{3} = \begin{bmatrix} 5 \\ 3 \\ 0.5 \end{bmatrix} \,,
\end{align}
written in their \emph{homogeneous} form.
What are the corresponding inhomogeneous coordinates \(\vect{v}_i\)?
\solution
The four 2D points are given in standard coordinates
\begin{align}
\vect{v}_{0} = \begin{bmatrix} 1 \\ 2 \end{bmatrix} \,, \;
\vect{v}_{1} = \begin{bmatrix} 2 \\ 1 \end{bmatrix} \,, \;
\vect{v}_{2} = \begin{bmatrix} -6 \\ -4 \end{bmatrix} \,, \AND
\vect{v}_{3} = \begin{bmatrix} 10 \\ 6 \end{bmatrix} \,.
\end{align}
\end{exercise}

\begin{exercise}
Consider now instead another four three-dimensional (3D) points
\begin{align}
\vect{Q}_{0} = \begin{bmatrix} 1   \\ 10 \\ -3  \\ 1 \end{bmatrix} \,, \;
\vect{Q}_{1} = \begin{bmatrix} 2   \\ -4 \\ 1.1 \\ 2 \end{bmatrix} \,, \;
\vect{Q}_{2} = \begin{bmatrix} 0   \\ 0  \\ -1  \\ 10 \end{bmatrix} \,, \AND
\vect{Q}_{3} = \begin{bmatrix} -15 \\ 3  \\ 6   \\ 3 \end{bmatrix} \,.
\end{align}
What are then the corresponding inhomogeneous coordinates \(\vect{V}_i\)?
\solution
The four 2D points are given in standard coordinates
\begin{align}
\vect{V}_{0} = \begin{bmatrix} 1   \\ 10 \\ -3 \end{bmatrix} \,, \;
\vect{V}_{1} = \begin{bmatrix} 1   \\ -2 \\ 0.55 \end{bmatrix} \,, \;
\vect{V}_{2} = \begin{bmatrix} 0   \\ 0  \\ -0.1 \end{bmatrix} \,, \AND
\vect{V}_{3} = \begin{bmatrix} -5 \\ 1  \\ 2 \end{bmatrix} \,.
\end{align}
\end{exercise}

\begin{exercise}\label{exc:linedef}
A 2D line is given as
\begin{align}
x + 2 y = 3 \,.
\end{align}
Write this line in homogeneous form i.e. \(\vect{l}^{T} \vect{q} = 0\) where \(\vect{q} = {[x\, y\, w]}^\text{T}\).
\solution
The 2D line \(\vect{l}\) where \(\vect{l}^\text{T} \vect{q} = 0\) and \(\vect{q} = {[x\, y\, w]}^\text{T}\) is
\begin{align}
\vect{l} = \begin{bmatrix} 1 \\ 2  \\ -3 \end{bmatrix}
\end{align}
\end{exercise}

\begin{exercise}
Using the same line definition \(\vect{l}\) as in \cref{exc:linedef}, which of the following 2D points are on this line?
\begin{align}
\vect{q}_{0} = \begin{bmatrix} 3   \\ 0   \\ 1  \end{bmatrix} \,, \;
\vect{q}_{1} = \begin{bmatrix} 6   \\ 0   \\ 2  \end{bmatrix} \,, \;
\vect{q}_{2} = \begin{bmatrix} 1   \\ 1   \\ 2  \end{bmatrix} \,, \;
\vect{q}_{3} = \begin{bmatrix} 110 \\ -40 \\ 10 \end{bmatrix} \,, \AND
\vect{q}_{4} = \begin{bmatrix} 11  \\ 4   \\ 1  \end{bmatrix} \,.
\end{align}
\solution
Any point on the line must obey \(\vect{l}^\text{T} \vect{q} = 0\). Using the same line \(\vect{l}\) as defined in \cref{exc:linedef} we find
\begin{align}
\vect{l}^\text{T}\vect{q}_{0} = 0 \,, \; \;
\vect{l}^\text{T}\vect{q}_{1} = 0 \,, \; \;
\vect{l}^\text{T}\vect{q}_{2} = -3\,, \; \;
\vect{l}^\text{T}\vect{q}_{3} = 0 \,, \AND \;
\vect{l}^\text{T}\vect{q}_{4} = 16 \,.
\end{align}
In other words, points \(\vect{q}_{0}\), \(\vect{q}_{1}\), and \(\vect{q}_{3}\) are all on the line \(vectsymbf{l}\); the points \(\vect{q}_{2}\) and \(\vect{q}_{4}\) are not.
\end{exercise}

\begin{exercise}
Given the two lines
\begin{align}
\vect{l}_{0} = \begin{bmatrix} 1  \\ 1 \\ -1 \end{bmatrix} \, \AND
\vect{l}_{1} = \begin{bmatrix} -1 \\ 1 \\ -3 \end{bmatrix} \,,
\end{align}
what is their point of intersection \(\vect{q}_\textup{I}\)?
\solution
  The intersection of two lines is the cross product of the homogeneous coordinate definitions.
  \begin{align}
    \vect{q}_\textup{I} = \vect{l}_{0} \times \vect{l}_{1} = \begin{bmatrix} -2 \\ 4 \\ 2 \end{bmatrix} \,.
  \end{align}
\end{exercise}

\begin{exercise}
Consider now the matrix
\begin{align}
\matr{A} = \begin{bmatrix}
             10 &  0 &  2 \\
               0 & 10 & -3 \\
               0 &  0 &  1
           \end{bmatrix} \,.
\end{align}
What is the result of \(\matr{A}\vect{q}\) (\(\matr{A}\) ``dot'' \(\vect{q}\)), where \(\vect{q}\) is a 2D point in homogeneous coordinates?\\
Explain what each of the (nonzero) coefficients in \(\matr{A}\) does to the coordinates of \(\vect{q}\).
\solution
  The direct solution is \(\matr{A}\vect{q} = {[10 x + 2, 10 y - 3, 1]}\).
  The two 10's scale the x- and y-coordinates.
  If the last diagonal factor was 10 and not 1, this would have no effect on the inhomogeneous coordinates.
  The two remaining factors are 2D translations.
  Finally, also notice that the translations happened after scaling.
\end{exercise}

\begin{exercise}\label{ex:distdef}
A new line is given
\begin{align}
\vect{l} = \begin{bmatrix} \frac{1}{\sqrt{2}} \\ \frac{1}{\sqrt{2}} \\ -1 \end{bmatrix} \,.
\end{align}
What is the \emph{shortest distance} \(d_{i}\) between the line \(\vect{l}\) and the points
\begin{align}
\vect{q}_{0} = \begin{bmatrix} 0 \\ 0 \\ 1 \end{bmatrix} \,, \;
\vect{q}_{1} = \begin{bmatrix} \sqrt{2} \\ \sqrt{2} \\ 1 \end{bmatrix} \,, \AND
\vect{q}_{2} = \begin{bmatrix} \sqrt{2} \\ \sqrt{2} \\ 4 \end{bmatrix} \,?
\end{align}
\solution
The shortest distance \(d\) between a line \(\vect{l} = {[l_x, l_y, l_w]}^\text{T}\) and a point \(\vect{q}\) is given
\begin{align}
  d = \frac{\left|\vect{l}^\text{T} \vect{q}\right|}{|p_w| \sqrt{l_x^2 + l_y^2}} \,.
\end{align}
The solutions are
\begin{align}
  d_0 = 1 \,, \;
  d_1 = 1 \,, \AND
  d_2 = 1/2 \,.
\end{align}
\end{exercise}

\begin{exercise}
Repeat \cref{ex:distdef} with
\begin{align}
\vect{l} = \begin{bmatrix} 2 \\ 2 \\ -1 \end{bmatrix} \,.
\end{align}
\solution
The definition of the shortest distance \(d\) is given in \cref{ex:distdef}, and the new solutions are
\begin{align}
  d_0 = \frac{1}{\sqrt{8}} = 0.354 \,, \;
  d_1 = \frac{4\sqrt{2} - 1}{\sqrt{8}} = 1.646 \,, \AND
  d_2 = \frac{\sqrt{2} - 1}{\sqrt{8}} = 0.146 \,.
\end{align}
\end{exercise}

\begin{exercise}
A plane is given
\begin{align}
\vect{P} = \begin{bmatrix} 2 \\ 1 \\ 2 \\ 4 \end{bmatrix} \,.
\end{align}
What is the shortest distance \(D_{i}\) between the plane \(\vect{q}\) and the points
\begin{align}
\vect{Q}_{0} = \begin{bmatrix} 0 \\ 0 \\ 0 \\ 1 \end{bmatrix} \,, \;
\vect{Q}_{1} = \begin{bmatrix} \sqrt{2} \\ \sqrt{2} \\ 0 \\ 1 \end{bmatrix} \,, \AND
\vect{Q}_{2} = \begin{bmatrix} \sqrt{2} \\ \sqrt{2} \\ 0 \\ 4 \end{bmatrix} \,?
\end{align}
\solution
The shortest distance \(d\) between a plane given by \(\vect{P} = {[P_x, P_y, P_z, P_w]}^\text{T}\) and a point \(\vect{Q}\) is
\begin{align}
  d = \frac{\left|\vect{P}^\text{T} \vect{Q}\right|}{|Q_w| \sqrt{P_x^2 + P_y^2 + P_z^2}} \,.
\end{align}
The solutions are
\begin{align}
  d_0 =  1.333 \,, \;
  d_1 = 2.748 \,, \AND
  d_2 = 1.687 \,.
\end{align}
\end{exercise}

\begin{exercise}
  In computer vision, the most vital component is the camera.
  We need to be able to take pictures and analyze these pictures in a suitable programming language/tool.

  With your weapon of choice---Matlab, python, C/C++/C\#---create some code to take images and load them into the computer.
  Most of your computers should have a webcam you can use. You can use OpenCV in python or C/C++ while Matlab has their own tools.
  Build your own tool that can
  \begin{itemize}
    \item take an image,
    \item load the image into memory, and
    \item (optionally) show the image.
  \end{itemize}
  As an additional exercises, try to read the embedded EXIF data.
  What type of information is embedded here?

  If you really want a challenge, then try to use your phone. Either develop an application that you run on your phone, or somehow establish a connection to the phone camera.
  You will need previous experience with developing phone applications or connections.

  Please note that this exercise has no solution, and we cannot help you that much.
  This is an exercise for you to get started with building computer vision programs, and to find you own solutions using all the online resources that are available.
\end{exercise}

\subsection{Solutions}
\printallsolutions\

\end{exercises}

%\cleardoublepage\printbibliography\

\end{document}
